\documentclass{article}
\usepackage[finnish]{babel}
\usepackage[utf8]{inputenc}
%\usepackage{times}
\usepackage{authordate1-4}

\def\citename#1{#1}

\def\bibtitle{Viitteet}
\begin{document}


\title{Työsuunnitelma: Fenfire-järjestelmän jatkokehitys}
\author{Tuomas Lukka}
\date{}

\maketitle

\section{Taustaa}

Olemme kehittäneet Fenfire-järjestelmää Jyväskylän Yliopistossa,
ensin tietotekniikan laitoksella ja sittemmin Agora Centerissä
vuoden -99 syksystä alkaen\footnote{
Tavara\-merkki\-syistä nimi on muuttunut matkan varrella: projekti aloitettiin 
nimellä GZigZag \cite{lukka99cybertext}, 
sai vuonna -02
väliaikaisesti nimen Gzz ja keväästä -03 alkaen lopullisen uuden
nimen Fenfire} \cite{lukka99cybertext}.
Fenfire-projektin tavoitteena on suunnitella tietokoneen 
käyttö\-jär\-jes\-tel\-mä ja käyttöliittymät kokonaisvaltaisesti uusiksi,
ottaen paremmin huo\-mioon ihmisen psykologia sekä käyttöliittymissä
että esim.\ luotettavuudessa \cite{fallenstein03storm}.

Projekti koostuu keskeisestä, yhteisestä pitkän tähtäimen
visiosta siitä, millaista tietokoneen käytön tulisi olla,
pohjautuen esimerkiksi
Bushin \shortcite{as-we-may-think}, Engelbartin \shortcite{engelbart63conceptual-framework-augmenting-mans-intellect}
ja Nelsonin \shortcite{as-we-will-think} unelmiin tietokoneiden käyttäjilleen tarjoamista
mahdollisuuksista.
Tästä visiosta kumpuaa selkeitä
lyhyen tähtäimen tavoitteita jotka ovat mielenkiintoisia 
sekä tutkimuksellisesti että teknisesti koska ne poikkeavat voimakkaasti
valtavirrasta.



\section{Nykytilanne}

Syksyllä -03 valmistui
Fenfire-järjestelmän ensimmäinen konkreettinen osaprototyyppi, FenPDF, 
joka on suunnattu akateemisille tutkijoille joiden täytyy pystyä käsittelemään
alojensa valtavaa, koko ajan kasvavaa kirjallisuutta (josta suuri osa on saatavissa
verkosta PDF-tiedostoina, siksi nimi). 
Prototyyppi antaa mahdollisuuden
rakentaa selkeä struktuuri artikkeleista antamalla käyttäjälle
mahdollisuus tehdä transkluusioita \cite{ted-xanalogical-structure-needed}
PDF-tiedostoista, eli kopioida artikkelista palanen johonkin toiseen yhteyteen
kuitenkin siten, että yhteys alkuperäiseen kokonaiseen artikkeliin säilyy kaksisuuntaisesti.
Näin on mahdollista kopioida kaikki työstettävänä olevan artikkelin viitteiden
relevantit kappaleet yhteen paikkaan; kopioista voi suoraan lukea ja muistaa, 
mikä oli viitteen tärkein sanoma. Toisaalta myöhemmin luettaessa yhtä viitteistä
linkki työstettyyn artikkeliin on olemassa; on helppoa päästä käsiksi
muihin asiaan liittyneisiin artikkeleihin. 

FenPDF on tutkimusryhmämme
sisäisessä tuotantokäytössä ja on koettu erittäin hyödylliseksi 
alaamme liittyvän laajan kirjallisuuden hallitsemiseksi.

FenPDF sisältää useita erit\-täin radikaaleja käyttöliittymäratkaisuja,
esim.\ \cite{lukka02fillets,kujala03paper},
jotka näyttävät toimivan yhteen erin\-omaisesti ja luovat yhdessä
kokonaan uudenlaisen ym\-pä\-ris\-tön tietokoneen käyt\-tä\-mi\-seen.
Useita tärkeitä innovaatioita ei vielä ole julkaistu, esimerkiksi sitä 
miten linkit eivät korvaa koko ikkunan sisältöä käyttäjää hämäten vaan
linkatut asiat kelluvat marginaalissa ja koko linkkisiirtymä on käyttäjän
näkökulmasta sulava ja helposti käännettävä.

Projektin pitkä\-jän\-tei\-syy\-den
vuoksi
projektin julkaisutoiminta on vasta käyn\-nistymässä. 
Tähän mennessä projekti on tuottanut neljä konferenssiartikkelia,
joista kaksi erittäin korkeatasoisessa ACM Hypertext -konferenssissa.
Useat kehittelemistämme ideoista alkavat vasta nyt olla kypsiä julkaistavaksi
journal-artik\-keleissa, koska ideat todella poikkeavat paljon valtavirrasta
ja niiden kehittelyyn on siksi mennyt paljon aikaa.


\section{Tavoitteet}

Tämän osaprojektin tavoitteena on 
sekä viimeistellä FenPDF-työkalua julkista levitystä varten
että kehittää edelleen Fenfiren keskeisiä
ajatuksia ja aloittaa seuraavia osaprojekteja. Yksi tärkeä tutkimuskysymys
on, miten muut tietomallit (esim.\ sähköposti) saadaan nivoutumaan FenPDF:ään
saumattomasti.

Lisäksi FenPDF:n radikaalit käyttöliittymäratkaisut ja etenkin niitten toi\-mi\-vuus
ovat avanneet uusia perustutkimuksellisia kysymyksiä - alustavat 
käyttäjä\-testit FenPDF:llä ovat osoittaneet, että näissä uusissa 
ratkaisuissa on selvästi jotakin erityislaatuista, ja on odotettavissa, että tarkempi selvitys 
käytettävyys\-tut\-ki\-muk\-sel\-la voi johtaa huomattaviin uusiin tuloksiin.
Näissä kysymyksissä
tutkimusryhmämme on aloittanut yhteistyön prof.~Pertti Saariluoman kanssa


\section{Merkitys}

FenPDF-ohjelmiston kehityksen avaamat perustutkimukselliset kysymykset
ovat erit\-täin mielenkiintoisia koska ne tuovat uusia näkökulmia
havaintopsykologisiin kysymyksiin, esimerkiksi värien näkemiseen.


Tietokoneiden käyttöliittymät ovat muuttuneet viimeisen 20 vuoden ku\-lu\-es\-sa
vain hyvin vähän, vaikka koneiden teho on kasvanut rä\-jäh\-dys\-mäi\-ses\-ti.
Nykyiset järjestelmät vaativat ihmisiä tekemään ja ajattelemaan asioita
tietyllä tavalla, lokeroiden käyttäjälle tärkeän tiedon epäluonnollisesti.
Jotta tietotekniikasta voitaisiin tehdä ihmisille oikeasti hyödyllinen
apuväline eikä pelkkä monimutkainen kirjoituskone, on välttämätöntä tutkia juuri näitä asioita.

Tutkimustyön sivutuotteina syntyvät ohjelmistot ovat vapaasti levitettäviä
ja niiden tarkoitus on hyödyttää kaikkia tietokoneen käyttäjiä.


\section{Toteuttamissuunnitelma}

Tutkimus toteutetaan Jyväskylän Yliopiston Agora Centerissä.


\section{Kustannusarvio ja rahoitussuunnitelma}

(sivukustannukset pyöristetty lähimpään 50 euroon)

\begin{tabular}{l|l|r|r}
Kohde & Tarkennus &  & Summa\\
\hline
Henk.koht apuraha & & & 4000 \\
Aputyövoima & & & 12000 \\
& Benjamin Fallenstein, osa-aikainen palkka 10kk & 10250 \\
& Sivukustannukset 17\%        & 1750 \\
Laitteet                & & & 2000 \\
& Kannettava tietokone & 1500 \\
& Ohjaimia, Java-kännykkä, ... & 500 & \\
\hline
Yhteensä & & & 20000 \\
\hline
\multicolumn{3}{l}{Rahoitus Suomen kulttuurirahastolta tähän projektiin}
    & $-20000$ \\
\end{tabular}



\bibliographystyle{authordate1}
\bibliography{gzigzag}

\end{document}
